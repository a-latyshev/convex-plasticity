\documentclass[12pt]{article}

% \usepackage[latin1]{inputenc}    
\usepackage[T1]{fontenc}
\usepackage[french]{babel} 
\usepackage[utf8]{inputenc} 
\usepackage{amsmath, amssymb}
\usepackage[top=2.5cm, bottom=2.5cm, left=2cm, right=2cm]{geometry}
\usepackage{graphicx}
\usepackage{float}
\usepackage{titlesec}
\usepackage{gensymb}
\usepackage{hyperref}

\usepackage{eurosym}

\usepackage{setspace} \onehalfspacing % interline interval

\newcommand\ul[1]{\underline{#1}}
\newcommand\uusigma{\ul{\ul{\sigma}}}
\newcommand\uv{\ul{v}}
\newcommand\uF{\ul{F}}
\newcommand\uu{\ul{u}}
\newcommand\un{\ul{n}}
\newcommand\uus{\ul{\ul{s}}}
\newcommand\uue{\ul{\ul{e}}}
\newcommand\uuI{\ul{\ul{I}}}
\newcommand\uuUnit{\ul{\ul{1}}}
\newcommand\uueps{\ul{\ul{\varepsilon}}}
\newcommand{\uuuuK}{\ul{\ul{\ul{\ul{K}}}}}
\newcommand{\uuuuJ}{\ul{\ul{\ul{\ul{J}}}}}
\newcommand{\uuuuC}{\ul{\ul{\ul{\ul{C}}}}}
\newcommand{\uuuuD}{\ul{\ul{\ul{\ul{D}}}}}

\newcommand\Deltaepsp{\Delta \uueps_{n}^p}
\newcommand\Deltaeps{\Delta \uueps_{n}}
\newcommand\Deltaep{\Delta \uue_{n}^p}

\newcommand\sigmaeqn{\sigma_{n+1}^\text{eq}}
\newcommand\sigmaeqnelas{\sigma_{n+1}^\text{elas,eq}}
\newcommand\uusn{\uus_{n+1}}
\newcommand\uusnelas{\uus_{n+1}^\text{elas}}
\newcommand\uusigman{\ul{\ul{\sigma}}_{n+1}}
\newcommand\uusigmanelas{\uusigman^{\text{elas}}}
\newcommand{\uunelas}{\ul{\ul{n}}_\text{elas}}
\newcommand\mtr{\mathrm{tr}}
\newcommand\md{\mathrm{d}}


\begin{document}

\begin{titlepage}
	% \thispagestyle{empty}
	\newcommand{\HRule}{\rule{\linewidth}{0.5mm}}
	\center
	% \textsc{\large ECOLE NATIONALE DES PONTS ET CHAUSSEES}\\[.7cm]
	% \includegraphics[width=35mm]{../img/img/ENPC_logo.png}\\[.5cm]
	% \textsc{École des Ponts ParisTech}\\[.5cm]
    % \textsc{2020-2021}\\[0.5cm]
    
	\textsc{\large Rapport de stage PFE}\\[0.5cm]
    \textsc{\large Andrey Latyshev}\\[0.5cm]    
    \textsc{Élève ingénieur de double diplôme}\\[1.5cm]

    % \textsc{\large Stage 1 : Modélisation numérique des joints poreux }\\
	% \textsc{\large Stage 2 : Estimation de l'amortissement en roulis d'un corps flottant par méthode numérique (CFD) }\\[1.5cm]
    % \textsc{\small Stage 1 réalisé au sein du Laboratoire Navier, CNRS}\\[0.1cm]
    % \textsc{\small 6 et 8 avenue Blaise Pascal, Champs-sur-Marne, 77455 Marne la Vallée cedex 2}\\[0.1cm]
    % \textsc{\small Dates du stage : 08/06/2020 - 11/09/2021}\\[1cm]

    % \textsc{\small Stage 2 réalisé au sein de Doris Engineering}\\[0.1cm]
    % \textsc{\small 58a Rue du Dessous des Berges, 75013, Paris}\\[0.1cm]
    % \textsc{\small Dates du stage : 14/09/2020 - 14/02/2021}\\[1cm]

	% \textsc{\large Maître de stage 1 : M. Patrick Dangla}\\[0.5cm]
	% \textsc{\large Maître de stage 2 : M. Evgeny Andreev}\\[0.5cm]
	\textsc{\huge DRAFT}

\end{titlepage}

\pagestyle{plain}

\section{Critère de Drucker-Prager}

\begin{equation}
    f(\uusigma, p) = \sigma_{\text{eq}} + \alpha \mtr \uusigma - R(p) \leq 0
\end{equation}

\begin{equation}
    \dot{\uueps}^p = \dot{\lambda} \frac{\partial f}{\partial \uusigma} = \dot{\lambda} \left(\frac{3}{2} \frac{\uus}{\sigma_\text{eq}} + \alpha \uuUnit \right)
\end{equation}

\begin{align}
    & \dot{p} = \sqrt{\frac{2}{3} \dot{\uueps}^p : \dot{\uueps}^p} = \dot{\lambda}\sqrt{1 + 2\alpha^2} \\
    & \Deltaepsp = \frac{\Delta p_n}{\sqrt{1 + 2\alpha^2}} \left( \frac{3}{2} \frac{\uusn}{\sigmaeqn} + \alpha \uuUnit \right) \\
    & \Deltaep = \frac{\Delta p_n}{\sqrt{1 + 2\alpha^2}} \frac{3}{2} \frac{\uusn}{\sigmaeqn} 
\end{align}

\begin{align}
    & \uusigma = (3k\uuuuJ + 2\mu\uuuuK) : (\uueps - \uueps^p) = \uusigma_{n+1}^\text{elas} - (3k\uuuuJ + 2\mu\uuuuK) : \uueps^p
\end{align}

\begin{align}
    & \uusigma_{n+1} = \uusigma_{n+1}^\text{elas} - 2\mu \Deltaep - k \mtr (\Deltaepsp) \uuUnit \\
    & \uus_{n+1} = \uus_{n+1}^\text{elas} - 2\mu\Deltaep \\
    & \uus_{n+1}^\text{elas} = \uus_{n+1} + 3\mu\frac{\Delta p_n}{\sqrt{1 + 2\alpha^2}} \frac{\uusn}{\sigmaeqn} = \uus_{n+1} (1 +  3\mu\frac{\Delta p_n}{\sqrt{1 + 2\alpha^2}} \frac{1}{\sigmaeqn}) \\
    & \sigmaeqnelas = \sigmaeqn (1 +  3\mu\frac{\Delta p_n}{\sqrt{1 + 2\alpha^2}} \frac{1}{\sigmaeqn}) \\
    & \frac{\uusnelas}{\sigmaeqnelas} = \frac{\uusn}{\sigmaeqn}
\end{align}

\begin{align}
    & \sigmaeqn = \sigmaeqnelas - \frac{3\mu}{\sqrt{1 + 2\alpha^2}}\Delta p_n \\
    & \mtr \uusigman = \mtr \uusigmanelas - 3\kappa \mtr \Deltaepsp \\
    & \mtr \Deltaepsp = \frac{3\alpha}{\sqrt{1 + 2\alpha^2}} \Delta p_n
\end{align}

\begin{align}
    & \sigmaeqn + \alpha \mtr \uusigman - R(p_n + \Delta p_n) = 0 \\
    & R(p) = \sigma_0 + h p \\
    & \sigmaeqnelas - \frac{3\mu}{\sqrt{1 + 2\alpha^2}}\Delta p_n + \alpha \mtr \uusigmanelas - 3\kappa \frac{3\alpha^2}{\sqrt{1 + 2\alpha^2}} \Delta p_n - \sigma_0 - h p_n - h \Delta p_n = 0 \\
    & \Delta p_n = \frac{ \sigmaeqnelas + \alpha \mtr \uusigmanelas - \sigma_0 - h p_n}{ \frac{3\mu + 9\alpha^2\kappa}{\sqrt{1 + 2\alpha^2}} + h} = \frac{ \sigmaeqnelas + \alpha \mtr \uusigmanelas - \sigma_0 - h p_n}{ \gamma } \\
    & \gamma = \frac{3\mu + 9\alpha^2\kappa}{\sqrt{1 + 2\alpha^2}} + h
\end{align}

\begin{align}
    & \uusigma_{n+1} = \uusigma_{n+1}^\text{elas} - 2\mu \Deltaep - \kappa \mtr (\Deltaepsp) \uuUnit \\
    & \uusigma_{n+1} = \uusigma_{n+1}^\text{elas} - \frac{1}{\sqrt{1 + 2\alpha^2}}\left( \beta \uusnelas + 3k\alpha \Delta p_n \uuUnit \right) \\
    & \beta = 3\mu\frac{\Delta p_n}{\sigmaeqnelas} 
\end{align}

\begin{align}
    & \frac{\partial \uusnelas{}}{\partial \Deltaeps} = 2\mu \uuuuK\\
    & \frac{\partial \sigmaeqnelas{}}{\partial \Deltaeps} = \frac{3\mu}{\sigmaeqnelas}\uusnelas \\
    & \frac{\partial\mtr\sigmaeqnelas}{\partial\Deltaeps} = 3k\uuUnit \\
    % & \frac{\partial (\mtr (\Deltaepsp) )}{\partial \Deltaeps} = \uuUnit \\
    % & \frac{\partial \Deltaepsp}{\partial \Deltaeps} = \frac{\partial}{\partial \Deltaeps} \left( \frac{\Delta p_n}{\sqrt{1 + 2\alpha^2}} \left( \frac{3}{2} \frac{\uusn}{\sigmaeqn} + \alpha \uuUnit \right)  \right) \\
    & 2\mu\Deltaep + k\mtr\Deltaepsp\uuUnit = \frac{\Delta p_n}{\sqrt{1 + 2\alpha^2}} \left( 3\mu \frac{\uusnelas}{\sigmaeqnelas} + 3k\alpha \uuUnit  \right) \\
    & \frac{\partial\Delta p_n}{\partial\Deltaeps} = \frac{1}{\gamma} \frac{\partial(\sigmaeqnelas + \alpha \mtr \uusigmanelas)}{\partial\Deltaeps} = \frac{1}{\gamma} \left( \frac{3\mu}{\sigmaeqnelas}\uusnelas + 3k\alpha\uuUnit \right) = \frac{1}{\gamma}(3\mu \uunelas + 3k\uuUnit)
\end{align}

\begin{align}
    & \frac{\partial\uusigman}{\partial \Deltaeps} = \uuuuC - \frac{\partial (2\mu\Deltaep + k\mtr\Deltaepsp\uuUnit)}{\partial\Deltaeps} = \uuuuC - \uuuuD \\
    & \uuuuD = \frac{1}{\sqrt{1 + 2\alpha^2}} \frac{\partial }{\partial \Deltaeps} \left( \Delta p_n \left( 3\mu \frac{\uusnelas}{\sigmaeqnelas} + 3k\alpha \uuUnit  \right) \right) = \\
    & = \frac{1}{\sqrt{1 + 2\alpha^2}}\left( \frac{\partial\Delta p_n}{\partial\Deltaeps} \otimes \left( 3\mu\frac{\uusnelas}{\sigmaeqnelas} + 3k\alpha \uuUnit \right) + \Delta p_n \left(3\mu\frac{\partial\uusnelas}{\partial\Deltaeps}\frac{1}{\sigmaeqnelas} \right) - \Delta p_n 3\mu\frac{\uusnelas}{(\sigmaeqnelas)^2} \otimes \frac{\partial\sigmaeqnelas}{\partial\Deltaeps}\right) = \\
    & = \frac{1}{\sqrt{1 + 2\alpha^2}}\left( \left( 3\mu \uunelas + 3\alpha k \uuUnit \right) \frac{1}{\gamma} \otimes \left( 3\mu \uunelas + 3\alpha k \uuUnit \right) + 2\mu\beta\uuuuK - 3\mu\beta\uunelas \otimes \uunelas \right) \\
    & \uuuuD = \frac{1}{\sqrt{1 + 2\alpha^2}}\left( 3\mu\left(\frac{3\mu}{\gamma} - \beta\right)\uunelas \otimes \uunelas + \frac{9\alpha\mu k}{\gamma} (\uuUnit \otimes \uunelas + \uunelas \otimes \uuUnit) + \frac{9\alpha^2k^2}{\gamma} \uuUnit \otimes \uuUnit + 2\mu\beta \uuuuK \right) \\
    % & \uuuuD : \Deltaeps = \frac{1}{\sqrt{1 + 2\alpha^2}}\left( \uunelas : \Deltaeps \left( \frac{9\alpha\mu k}{\gamma}\uuUnit + 3\mu\left(\frac{3\mu}{\gamma} - \beta\right)\uunelas \right) + \mtr \Deltaeps \left( \frac{9\alpha\mu k}{\gamma}  \uunelas + \frac{9\alpha^2k^2}{\gamma} \uuUnit \right) + 2\mu\beta \Deltaep \right) \\
    & \uuuuD : \Deltaeps = \\
    & = \frac{1}{\sqrt{1 + 2\alpha^2}}\left( \uunelas : \Deltaeps 3\mu \left( \frac{3\mu}{\gamma} - \beta \right) \uunelas + \frac{9\alpha\mu k}{\gamma} (\uunelas : \Deltaeps \uuUnit + \mtr\Deltaeps \uunelas) + \frac{9\alpha^2k^2}{\gamma} \mtr \Deltaeps \uuUnit + 2\mu\beta \Deltaep \right)  \\
\end{align}

\begin{align}
    & F = \int\limits_\Omega \uusigman : \uueps(\uv) \, \md\Omega - \uF_\text{ext} = \\ 
    & = \int\limits_\Omega \left( \uusigma_n + \uuuuC : (\Deltaeps - \Deltaeps^p) \right) : \uueps(\uv) \, \md\Omega - \uF_\text{ext} 
\end{align}
where $\uF_\text{ext} = q \int\limits_{\partial\Omega_\text{inside}}\un \cdot \uv \, \md s $, $\Deltaeps = \uueps(\Delta\uu_n)$ and $\Deltaeps = \uueps^p(\Delta\uu_n)$

\begin{align}
    & \uueps(\Delta\uu_n) = \frac{1}{2} \left( \nabla\Delta\uu + \nabla\Delta\uu^T \right) \\
    & \uueps^p(\Delta\uu_n) = 
        \begin{cases}
            \Delta p_n \left( \frac{3}{2}\frac{\uusnelas}{\sigmaeqnelas} + \alpha \uuUnit \right), & \text{ if } f(\uusigma^\text{elas}, p_n) > 0  \\
            0, & \text{ otherwise}
        \end{cases}
\end{align}
where $\Delta p_n = p_{n+1} - p_n$ 

\begin{align}
    & \uueps^p(\Delta\uu) = \uueps^p(\Delta\uu, p_n, p_{n+1}, \uusigma_n)
\end{align}

\begin{align}
    & F(\Delta\uu, \uv) = \int\limits_\Omega \left( \uusigma_n + \uuuuC : (\uueps(\Delta\uu) - \uueps^p(\Delta\uu)) \right) : \uueps(\uv) \, \md\Omega - \uF_\text{ext} \\
    & J(\uu, \uv) = \frac{\partial F(\Delta\uu, \uv)}{\partial\Delta\uu}(\uu)
\end{align}

Plan

\begin{enumerate}
    \item Introduction
    \item Main part
    \begin{enumerate}
        \item Theory
        \begin{enumerate}
            \item Plasticity and return-mapping algorithm
            \item Plasticity using conic optimization
            \item Plasticity problem description
        \end{enumerate}
        \item Development
        \begin{enumerate}
            \item Conventional return-mapping algorithm
            \item Custom assembler implementation in FEniCsX (motivs: convex + loi de comport.)
            \item Return-mapping via conic optimization
        \end{enumerate} 
        \item Results
        \begin{enumerate}
            \item Qualitative yield criterions comparison
            \item Calculation time for different approaches and solvers
            \item Patch size influence for vectorized convex optimization problem
            \item Different convex solvers comparison (+ patches)
        \end{enumerate}
        \item Perspectives*
        \begin{enumerate}
            \item Convex approach integration into the custom assembler
            \item Adaptation of diffcp library
        \end{enumerate}
    \end{enumerate}
    \item Conclusion
\end{enumerate}


\end{document}
