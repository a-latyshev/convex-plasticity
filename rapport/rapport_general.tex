\documentclass[12pt]{article}

% \usepackage[latin1]{inputenc}    
\usepackage[T1]{fontenc}
\usepackage[french]{babel} 
\usepackage[utf8]{inputenc} 
\usepackage{amsmath, amssymb}
\usepackage[top=2.5cm, bottom=2.5cm, left=2cm, right=2cm]{geometry}
\usepackage{graphicx}
\usepackage{float}
\usepackage{titlesec}
\usepackage{gensymb}
\usepackage{hyperref}


% \usepackage{apacite}
% \usepackage{listings}  
% \usepackage{color}
% \usepackage{stmaryrd}
% \usepackage{eurosym}
% \usepackage{natbib}
\usepackage[style=apa]{biblatex}
\addbibresource{src/bibs.bib}

\usepackage{setspace} \onehalfspacing % interline interval

% \renewcommand\refname{Bibliographie}
% \addto\captionsngerman{\renewcommand*{\bibname}{Literaturliste}}
% \addto\captionsrussian{\def\refname{Literaturliste}}
% \documentclass[sigconf,natbib=false]{acmart}
% \usepackage[
% backend=biber,
% style=apa,
% sorting=ynt
% ]{biblatex}
% \addbibresource{bibs.bib}

\setlength{\parindent}{1.25cm} 
\setlength{\parskip}{1em}

% footers and headers
\usepackage{fancyhdr}
\pagestyle{fancy} 
\fancyhead{}
\fancyfoot{}
\fancyhead[L]{Ecole Nationale des Ponts et Chaussées - Projet de fin d'Etudes}
\fancyfoot[R]{\thepage}
\fancyfoot[L]{Latyshev Andrey - Département Génie Mécanique et Matériaux}
\renewcommand{\headrulewidth}{0.0pt}

%Underlining ruler for subsections
% \titleformat{\section}
%   {\normalfont\Large\bfseries}
%   {\thesection}
%   {1em}
%   {#1}
% %   [{\titlerule[0.8pt]}]
% \titleformat{\section}
% {\normalfont\normalsize}
% {\uline{\thesection} }
% {1em}
% {\uline}
% \titlespacing{\section}
% {0pt}
% {\baselineskip}
% {0.4\baselineskip}

\usepackage{src/Latyshev_style}

\begin{document}

\begin{titlepage}
	{
        \center
	    \includegraphics[width=40mm]{img/ENPC_logo.png}\\[.44cm]
	    {\large École des Ponts ParisTech}\\[0.2cm]
	    {\normalsize 2021-2022}\\[.64cm]

	    {\Large Projet de Fin d'Etudes}\\[.5cm]
        {\large Département Génie Mécanique et Matériaux}\\[.85cm]
        {\Large Andrey Latyshev}\\[1cm]    
        {\large Élève ingénieur de double diplôme}\\[.85cm]

        {\Large Finite-element implementation of standard and softening plasticity using a convex optimization approach}\\[1cm]
        {\normalsize Projet réalisé au sein de Laboratoire Navier, ENPC}\\[0.2cm]
	    {\normalsize 6 et 8 avenue Blaise Pascal, Champs-sur-Marne, 77455}\\[0.2cm]
        {\normalsize 21/03/2022 - 09/09/2022}\\[1.1cm]

        {\large Tuteur: Matthieu Vandamme}\\[1.1cm]
    }
    \noindent \textbf{\normalsize Composition du jury}\\
    {\normalsize Président: Civilité Prénom Nom}\\
    {\normalsize Directeur de projet: Civilité Prénom Nom}\\
    {\normalsize Conseiller d'études: Civilité Prénom Nom}
\end{titlepage}

\section*{\centering Remerciement}
\setcounter{page}{2}

Je remercie Matthieu Vandamme pour son aide dans la recherche d'un stage, car en raison du début de la pandémie, la plupart des offres ont été fermées. Dans ces conditions, il était extrêmement difficile de trouver un premier stage. Sans son aide, il est peu probable que je commence l'expérience si tôt, ce qui était important pour mon cursus académique.

Je suis reconnaissant à Patrick Dangla et Siavash Ghabezloo pour leur mentorat et leurs conseils lors de mon premier stage dans le laboratoire de Navier. Cela m'a permis d'approfondir mes connaissances en mécanique des roches.

Je remercie également Evgeny Andreev et Olivier Langeard pour leur aide et leur travail commun. Grâce à eux, j'ai appris plus rapidement un nouveau domaine de la simulation de navires.

\newpage
\section*{\centering Abstract}
This internship aims at exploring a finite-element formulation of plasticity in the next generation FEniCS problem solving environment. The main goal is to propose an efficient and generic implementation which can tackle both classical and softening plasticity models. The intern will first familiarize himself with the DOLFINx computational environment and adapt existing implementations of legacy FEniCS code. The implementation will be then extended to the resolution of the local plasticity problem using convex optimization solvers and assess its efficiency compared to standard return mapping algorithms. Finally, softening plasticity will be considered and regularization strategies in order to prevent mesh dependency will be explored.

\newpage
\section*{\centering Résumé}

\newpage
\section*{\centering Synthèse du rapport en français}

\renewcommand{\listtablename}{\centering Liste des tableaux}
\renewcommand{\contentsname}{\centering Table des matières}
\renewcommand{\listfigurename}{\centering Liste des figures}
\newpage
\tableofcontents
\newpage
\listoftables
\newpage
\listoffigures
\newpage

\addcontentsline{toc}{section}{Introduction}
\section*{Introduction}

\newpage
% \addcontentsline{toc}{section}{Contexte et présentation de l’entreprise}
\section{Contexte et présentation de l’entreprise}

\newpage
\section{Revue de littérature}

\newpage
\section{Méthodologie}

\newpage
\section{Résultats}

\newpage
\section{Discussion}

sdf \parencite{Coussy} sdf \parencite{BRUNO2020724} sdf \parencite{Numba2015} sd \parencite{diamond2016cvxpy} s \parencite{bib:Domahidi2013ecos}

\newpage 
\addcontentsline{toc}{section}{Conclusion}
\section*{Conclusion}

\newpage
\renewcommand\refname{\centering Bibliographie}
% \printbibheading
\printbibliography
% \bibliographystyle{src/apa.bst}
% \bibliography{bibs}

\end{document}


% @article{BRUNO2020724,
%   title = {Return-mapping algorithms for associative isotropic hardening plasticity using conic optimization},
%   journal = {Applied Mathematical Modelling},
%   volume = {78},
%   pages = {724-748},
%   year = {2020},
%   issn = {0307-904X},
%   doi = {https://doi.org/10.1016/j.apm.2019.10.006},
%   url = {https://www.sciencedirect.com/science/article/pii/S0307904X19305943},
%   author = {Hugo Bruno and Guilherme Barros and Ivan F.M. Menezes and Luiz Fernando Martha}
% }