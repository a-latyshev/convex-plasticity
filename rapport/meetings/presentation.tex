\documentclass[9pt]{beamer}

% \usepackage{cmap}					% поиск в PDF
% \usepackage{mathtext} 				% русские буквы в формулах
% \usepackage[T2A]{fontenc}			% кодировка
\usepackage[T1]{fontenc}		% кодировка исходного текста
\usepackage[english]{babel}	% локализация и переносы    hyperref,,caption2
\usepackage[utf8]{inputenc}	
\usepackage{amsmath,amssymb,amsfonts,amsthm}
% \usepackage{graphicx}
% \usepackage{tikz}
% \usepackage[framemethod=TikZ]{mdframed} % For drawing frames and boxes\
\usepackage{minted}
\newmintinline[mypython]{python}{}
\newminted{python}{fontsize=\scriptsize, 
                   linenos,
                   numbersep=8pt,
                   gobble=4,
                   frame=lines,
                   bgcolor=bg,
                   framesep=3mm} 

\usepackage{listings}
% Define a custom color
\definecolor{backcolour}{rgb}{0.95,0.95,0.92}
\definecolor{codegreen}{rgb}{0,0.6,0}

% Define a custom style
% \lstdefinestyle{myStyle}{
%     backgroundcolor=\color{backcolour},   
%     commentstyle=\color{codegreen},
%     basicstyle=\ttfamily\footnotesize,
%     breakatwhitespace=false,         
%     breaklines=true,                 
%     keepspaces=true,                 
%     numbers=left,       
%     numbersep=5pt,                  
%     showspaces=false,                
%     showstringspaces=false,
%     showtabs=false,                  
%     tabsize=2,
% }

\usepackage{../Latyshev_style}

\definecolor{MidnightBlue}{rgb}{0.2,0.2,0.7}
\usetheme{Boadilla}
\setbeamertemplate{theorems}[numbered]
\setbeamertemplate{navigation symbols}{}
\hypersetup{pdfpagemode=FullScreen}
\setbeamertemplate{footline}
{
  \leavevmode%
  \hbox{%
  \begin{beamercolorbox}[wd=1\paperwidth,ht=2.25ex,dp=1ex,right]{date in head/foot}%
    \insertframenumber{} / \inserttotalframenumber\hspace*{2ex} 
  \end{beamercolorbox}}%
  \vskip0pt%
}

\title{Plasticity convex optimization}
\subtitle{Custom assembling of plasticity problem via FEniCSx 0.3.1.0}
\author[Latyshev A.]{Latyshev A. \\ [10mm]{\small Internship supervisors: \\ Jeremy Bleyer \\ Corrado Maurini}  }
\institute[Laboratoire Navier]{Laboratoire Navier}
\date{ENPC, May 16 2022} 

\begin{document}

\frame{\titlepage}

\begin{frame}[fragile]{Problem formulation}
  \begin{align}
    & f(\uusigma) = \sqrt{\frac{3}{2} \uus : \uus} - \sigma_0 - H p \leq 0 \\
    & \int_\Omega \left( \uuuuC^\text{tang} : \uueps(\ul{u})  \right) : \uueps(\ul{v}) dx + \int_\Omega \uusigma : \uueps(\ul{v}) dx + q\int_{\partial\Omega_{\text{inside}}}  \ul{n} \cdot \ul{v} dx = 0 \\
    & \uuuuC^\text{tang} = \uuuuC - 3\mu \left( \frac{3\mu}{3\mu + H} -\beta \right) \uun \otimes \uun - 2\mu\beta \ul{\ul{\ul{\ul{DEV}}}} \\
    & \Delta p = \frac{<f>_+}{3\mu + H}, \quad \beta = \frac{3\mu}{\sigma_\text{eq}}\Delta p, \quad \uun = \frac{\uus}{\sigma_\text{eq}} \\
    & \partial\Omega_\text{inside} : \uusigma \, \ul{n} = q\ul{n} \\
    & \partial\Omega_\text{left}, \partial\Omega_\text{right} : symmetry
  \end{align}

  \begin{pythoncode}
    a_Newton = ufl.inner(eps(v), inner_product(sig.tangent, eps(u_)))*dx
    res = -ufl.inner(eps(u_), as_3D_tensor(sig))*dx 
    res_Neumann = F_ext(u_)
  \end{pythoncode}
\end{frame}

\begin{frame}{FEniCSx entities and our needs}
  Fenics has following essential entities 
  \begin{itemize}
    \item \mintinline{python}{fem.Constants} (form constants). Size: number of constant scalars. FEniCS uses the same ensemble of values for every element, for every node in element
    \item \mintinline{python}{fem.Function} (form coefficients). Size: global number of DoFs * coefficient dimension $\approx$ number of cells * number of element nodes * coefficient dimension
    \item \mintinline{python}{fem.Expression}. Can be interpolated globally (only!)
  \end{itemize}

  Briefly. We need two inbetween entities:
  \begin{enumerate}
    \item \mintinline{python}{fem.Function} $\longleftrightarrow$ \mintinline{python}{fem.Expression}
    \item \mintinline{python}{fem.Constants} $\longleftrightarrow$ \mintinline{python}{fem.Function}
  \end{enumerate}
  The reason for this is two obstacles that we have encountered.

\end{frame}

\begin{frame}{Obstacle \#1}
  Let us consider the next simple variation problem
  \begin{equation}
    \int_{\Omega} g \cdot u v \, dx \quad \forall v \in V,
  \end{equation}
  where $g$ is a coefficient in Fenics therms.
  We want to update its global values locally, on the each assemble step, accordingly to a some expression like this one on which \mintinline{python}{fem.Expression} is based. At the same time UFL functionality is not sufficient for us. So we created a new entity named as \mintinline{python}{CustomFunction}. It 
  \begin{enumerate}
    \item inherits \mintinline{python}{fem.Function}
    \item contains a method \mintinline{python}{eval}, which will be called inside of the assemble loop and calculates the function local values
  \end{enumerate}
\end{frame}

\begin{frame}{Obstacle \#2}

  We want to economize memory avoiding a storage of all values of other fields. \mintinline{python}{fem.Constants} would be a good choice, but our field has different values on each element node. In plasticity problem terms it is $\uuuuC^\text{tang}$:
  \begin{itemize}
    \item sigma ($\uusigma$) is a coefficient with size: n\_cells * n\_gauss\_points * sigma\_dim
    \item C\_tang ($\uuuuC^\text{tang}$) is what ?? It has a size of 1 * n\_gauss\_points * C\_tang\_dim
  \end{itemize}

  So we need to allocate only (n\_gauss\_points * C\_tang\_dim) elements. We can cheat in Fenics using a field \mypython/x/ in the \mypython/fem.Function/ constructor. So we created a new entity \mypython/DummyFunction/. It 
  \begin{enumerate}
    \item inherits \mintinline{python}{fem.Function}
    \item has a size of 2 * n\_gauss\_points * C\_tang\_dim,
  \end{enumerate}
  where 2 is linked with the "minimal" triangulated square mesh.
\end{frame}

\begin{frame}[fragile]
  \frametitle{CustomFunction set up}
  We define \mypython/CustomFunction/ as follows
  \begin{pythoncode}
    sig = ca.CustomFunction(W, eps(Du), [C_tang, p, dp, sig_old], get_eval)
  \end{pythoncode}
  where p, dp, sig\_old variables have a type of \mypython/fem.Function/. Calculation of sig depends on these global fields. C\_tang is a \mypython/DummyFunction/. It should be recalculated on each element.
\end{frame}

\begin{frame}[fragile]{How to define eval method of CustomFunction}
  \begin{pythoncode}
    def get_eval(self:ca.CustomFunction):
    tabulated_eps = self.tabulated_input_expression
    n_gauss_points = len(self.input_expression.X)
    local_shape = self.local_shape
    C_tang_shape = self.tangent.shape
    
    @numba.njit(fastmath=True)
    def eval(sigma_current_local, sigma_old_local, p_old_local, dp_local, coeffs_values, constants_values, coordinates, local_index, orientation):
        deps_local = np.zeros(n_gauss_points*3*3, dtype=PETSc.ScalarType)
        
        C_tang_local = np.zeros((n_gauss_points, *C_tang_shape), dtype=PETSc.ScalarType)
        
        sigma_old = sigma_old_local.reshape((n_gauss_points, *local_shape))
        sigma_new = sigma_current_local.reshape((n_gauss_points, *local_shape))

        tabulated_eps(ca.ffi.from_buffer(deps_local), 
                      ca.ffi.from_buffer(coeffs_values), 
                      ca.ffi.from_buffer(constants_values), 
                      ca.ffi.from_buffer(coordinates), ca.ffi.from_buffer(local_index), ca.ffi.from_buffer(orientation))
        
        deps_local = deps_local.reshape((n_gauss_points, 3, 3))

        n_elas = np.zeros((3, 3), dtype=PETSc.ScalarType) 
        beta = np.zeros(1, dtype=PETSc.ScalarType) 
        dp = np.zeros(1, dtype=PETSc.ScalarType) 

        for q in range(n_gauss_points):
            sig_n = as_3D_array(sigma_old[q])
            sig_elas = sig_n + sigma(deps_local[q])
            s = sig_elas - np.trace(sig_elas)*I/3.
            sig_eq = np.sqrt(3./2. * inner(s, s))
            f_elas = sig_eq - sig0 - H*p_old_local[q]
            dp = ppos(f_elas)/(3*mu_+H)

            if f_elas >= 0:
                n_elas[:,:] = s/sig_eq*ppos(f_elas)/f_elas 
                beta[:] = 3*mu_*dp/sig_eq                 
            
            new_sig = sig_elas - beta*s
            sigma_new[q][:] = np.asarray([new_sig[0, 0], new_sig[1, 1], new_sig[2, 2], new_sig[0, 1]])
            dp_local[q] = dp
            
            C_tang_local[q][:] = get_C_tang(beta, n_elas)
        
        return [C_tang_local.flatten()] 
    return eval
  \end{pythoncode}
\end{frame}

\begin{frame}[fragile]{How to define the eval method of a CustomFunction}
  \begin{pythoncode}
    @numba.njit(fastmath=True)
    def local_assembling_b(cell, coeffs_values_global_b, coeffs_coeff_values_b, coeffs_dummy_values_b, coeffs_eval_b, u_local, coeffs_constants_b, geometry, entity_local_index, perm):
        sigma_local = coeffs_values_global_b[0][cell]
        p_local = coeffs_coeff_values_b[0][cell]
        dp_local = coeffs_coeff_values_b[1][cell]
        sigma_old_local = coeffs_coeff_values_b[2][cell]

        output_values = coeffs_eval_b[0](sigma_local, 
                                            sigma_old_local,
                                            p_local,
                                            dp_local,
                                            u_local, 
                                            coeffs_constants_b[0], 
                                            geometry, entity_local_index, perm)

        coeffs_b = sigma_local

        for i in range(len(coeffs_dummy_values_b)):
            coeffs_dummy_values_b[i][:] = output_values[i] #C_tang update

        return coeffs_b
  \end{pythoncode}

  \begin{pythoncode}
    @numba.njit(fastmath=True)
    def local_assembling_A(coeffs_dummy_values_b):
        coeffs_A = coeffs_dummy_values_b[0]
        return coeffs_A
  \end{pythoncode}

\end{frame}


\begin{frame}[fragile]{Custom assembling}
  We use FFCx/UFC approach to set up our own custom assembler.

  \begin{pythoncode}
    my_solver = ca.CustomSolver(a_Newton, res, Du, local_assembling_A, local_assembling_b, bcs)
  \end{pythoncode}

  \begin{pythoncode}
    for cell in range(num_owned_cells):
      pos = rows = cols = dofmap[cell]
      geometry[:] = coords[geo_dofs[cell], :]
      u_local = u[pos]
      x0_local = x0[pos]
      g_local = g[pos]

      b_local.fill(0.)
      A_local.fill(0.)

      coeffs_b = local_assembling_b(cell, coeffs_values_global_b, coeffs_coeff_values_b, coeffs_dummy_values_b, coeffs_eval_b, u_local, coeffs_constants_b, geometry, entity_local_index, perm)

      coeffs_A = local_assembling_A(coeffs_dummy_values_b)

      kernel_b(ffi.from_buffer(b_local), 
              ffi.from_buffer(coeffs_b),
              ffi.from_buffer(constants_values_b),
              ffi.from_buffer(geometry), ffi.from_buffer(entity_local_index), ffi.from_buffer(perm))
      
      kernel_A(ffi.from_buffer(A_local), 
              ffi.from_buffer(coeffs_A),
              ffi.from_buffer(constants_values_A),
              ffi.from_buffer(geometry), ffi.from_buffer(entity_local_index), ffi.from_buffer(perm))

      #Local apply lifting : b_local - scale * A_local_brut(g_local - x0_local)
      b_local -= scale * A_local @ ( g_local - x0_local )
      b[pos] += b_local

      MatSetValues_ctypes(A, N_dofs_element, rows.ctypes, N_dofs_element, cols.ctypes, A_local.ctypes, mode)
  \end{pythoncode}
\end{frame}

\begin{frame}[fragile]{Obstacle aka Neumann conditions}
  \begin{equation}
    a(u,v) = \int_\Omega... + ... + \int_\Omega... + \int_{\partial\Omega}... + ... + \int_{\partial\Omega}...
  \end{equation}
  \begin{pythoncode}
    def get_kernel(form:ufl.form.Form):
      domain = form.ufl_domain().ufl_cargo()
      ufcx_form, _, _ = ffcx_jit(domain.comm, form, form_compiler_params={"scalar_type": "double"})
      kernel = ufcx_form.integrals(0)[0].tabulate_tensor_float64
      return kernel
  \end{pythoncode}
  In fact this function gives us the first volume integral only. There are indexes (i)[j], where i represents the type on the integral (volume - 0, surface - 1) and j is the index of the integral in the sequence.

  Temporal solution:
  \begin{pythoncode}
    fem.petsc.assemble_vector(Res_Neumann, fem.form(res_Neumann))
    my_solver.assemble_vector(b_additional=Res_Neumann)
    ...
    if b_additional is not None:
    self.b.axpy(1, b_additional)
  \end{pythoncode}

\end{frame}

\begin{frame}[fragile]{Dirichlet conditions}
  \begin{pythoncode}
    self.g = fem.petsc.create_vector(self.b_form)
    self.x0 = fem.petsc.create_vector(self.b_form)
    ...
    if x0 is not None:
      fem.set_bc(self.x0, self.bcs, x0=self.g.array + x0, scale=-1.0)
    ...
    g_local = g[pos]
    x0_local = x0[pos]
    ...
    b_local -= scale * A_local @ ( g_local - x0_local )
  \end{pythoncode}
\end{frame}

\begin{frame}{Results. Plasticity}
  \begin{itemize}
    \item u, Du, sig, p, dp, sig\_old --- 6 global fields
    \item C\_tang --- 1 DummyFunction with the size of 2 * n\_gauss\_points * C\_tang\_dim 
    \item self.g, self.x0, b\_additional, dofmap\_topological --- 3 additional global fields and a full dofmap
  \end{itemize} 

\end{frame}

\begin{frame}{To Do}
  Tasks
  \begin{enumerate}
    \item Parallelization test
    \item "Neumann conditions" (can be omitted for now)
    \item Refactoring + cleaning 
    \item Memory freeing up  (can be omitted for now)
    \item Provide separated assembling (+ global apply lifting)
    \item Number of Newton iterations is greater than the one of classical algo.
  \end{enumerate}

  Refactoring
  \begin{itemize}
    \item \mypython/CustomFunction/ vs \mypython/CustomExpression/
    \item \mypython/DummyFunction/ $\rightarrow$ \mypython/AlteratedConstant/, \mypython/FunctionalConstant/, \mypython/RegularFunction/, \mypython/ConstantFunction/ ?
    \item \mypython/CustomSolver/ $\rightarrow$ \mypython/CustomProblem/
  \end{itemize}  

\end{frame}


% 	\begin{align*}
% 		\Gamma^{\text{bottom}} \cup \Gamma^{\text{top}} : \; & \uusigma \cdot \ul{n} = \ul{\sigma}_v, \quad p = P \\
% 		\Gamma^{\text{left}} \cup \Gamma^{\text{right}} : \; & \uusigma \cdot \ul{n} = \ul{\sigma}_h, \quad p = P 
% 	\end{align*}

% 	\begin{equation*}
% 		P(t) = 
% 			\begin{cases}
% 				0, & 0 < t \leq t_1 \\
% 				20 \cdot 10^6 \left( \frac{t - t_1}{t_2 - t_1} \right),  & t_1 < t \leq t_2
% 			\end{cases}
% 	\end{equation*}

% 	\begin{align*}
% 		& \sigma_h = -20 \text{ MPa} \\
% 		& \sigma_v = -30 \text{ MPa}
% 	\end{align*}
% \end{frame}

% \begin{frame}{Résultats}
% 	\begin{figure}[H]
% 		\centering
% 		\begin{minipage}[h]{0.49\linewidth}
% 			\center{\includegraphics[width=1\linewidth]{img/carre_pression.png} Pression \\ }
% 		\end{minipage}
% 		\hfill
% 		\begin{minipage}[h]{0.49\linewidth}
% 			\center{\includegraphics[width=1\linewidth]{img/carre_flow.png} Fluide flow \\ }
% 		\end{minipage}
% 		% \caption{Yield function}
% 	\end{figure}
% \end{frame}

% % \begin{frame}{Résultats}
% % 	\begin{figure}[H]
% % 		\centering
% % 		\begin{minipage}[h]{0.49\linewidth}
% % 			\center{\includegraphics[width=1\linewidth]{img/carre_pression.png} Pression \\ }
% % 		\end{minipage}
% % 		\hfill
% % 		\begin{minipage}[h]{0.49\linewidth}
% % 			\center{\includegraphics[width=1\linewidth]{img/carre_flow.png} Fluide flow \\ }
% % 		\end{minipage}
% % 		% \caption{Yield function}
% % 	\end{figure}
% % \end{frame}

% \begin{frame}{Résultats}
% 	\begin{figure}[H]
% 		\centering
% 		\begin{minipage}[h]{0.49\linewidth}
% 			\center{\includegraphics[width=1\linewidth]{img/carre_yield1.png} Étape 1 \\ }
% 		\end{minipage}
% 		\hfill
% 		\begin{minipage}[h]{0.49\linewidth}
% 			\center{\includegraphics[width=1\linewidth]{img/carre_yield2.png} Étape 2 \\ }
% 		\end{minipage}
% 		\caption{Yield function}
% 	\end{figure}
% \end{frame}


\end{document}
